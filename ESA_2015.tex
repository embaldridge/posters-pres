\documentclass{beamer}
\usetheme{Pittsburgh}
\usepackage{xcolor}
\definecolor{USUBlue}{RGB}{26,57,89}
\usecolortheme[named=USUBlue]{structure}
\setbeamercolor{normal text}{fg=USUBlue}
\usepackage[utf8]{inputenc}
\usepackage{mathptmx}
\usepackage{tgbonum}
\usepackage{amssymb}
\usepackage{graphicx}
\usepackage{marvosym}
\usepackage[absolute, overlay]{textpos}
\usefonttheme{structuresmallcapsserif}
\usefonttheme{serif}
\setbeamercolor{author}{fg=USUBlue}
\setbeamerfont{author}{size=\small}
\setbeamerfont{frametitle}{size=\large}
\setbeamertemplate{enumerate items}[default]
\author{Elita Baldridge, Department of Biology and the Ecology Center, Utah State University, Logan, UT, USA, @elitabaldridge\\Ethan White, Department of Wildlife Ecology \& Conservation and the Informatics Institute, University of Florida, Gainesville, FL, USA}
\title[17pt]{Ecologist in silico: Facilitating access for chronically ill/disabled ecologists}
\date{}
\setbeamertemplate{navigation symbols}{}

\usepackage[orientation=landscape,size=a0,scale=1.4,debug]{beamerposter}
 
\begin{document}
\begin{center} 
\begin{huge}
\rule{\linewidth}{3cm}
\textsc{%textsc makes text small caps
\\Ecologist in silico: Facilitating access for chronically ill/disabled ecologists\\
 }
\end{huge}  
\begin{large}
\textsc{Elita Baldridge}, Department of Biology and the Ecology Center, Utah State University, Logan, UT, USA, @elitabaldridge\\  
\textsc{Ethan White}, Department of Wildlife Ecology \& Conservation and the Informatics Institute, University of Florida, Gainesville, FL, USA, @ethanwhite\\
\end{large}
\end{center}
\begin{minipage}{0.25\linewidth}
\hspace{0.5cm}
\begin{Large}
\begin{center}
\textsc{Background}
\end{center}
\end{Large}
% Generalize language.
% Condense
Members of under-represented groups face unconscious and conscious biases which create societal barriers to doing science, but chronically ill/disabled scientists in particular often face physical, as well as societal, barriers to pursuing science.\\ 

\textsc{Disabilities can be visible or invisible, mental or physical, present with constant severity, get worse over time, or fluctuate from bad to less bad.\\ } 

Increasing accessibility for chronically ill/disabled ecologists also supports accessibility for ecologists who have traditionally had difficult accessing ecology for other reasons (lack of funds, distance, etc.). I present recommendations to encourage ecology to become a more accessible discipline to those with chronic illness/disability, from a general perspective and on the scale of individual collaborations. 
\begin{Large}
\begin{center}
\textsc{The Bare Minimum}
\end{center}
\end{Large}
\begin{center}
\includegraphics[scale=1]{../sad-comparison/sad-data/dissertation/trinket-access-generator.png}
\end{center}
\end{minipage}
\begin{minipage}{0.42\linewidth}
\begin{Large}
\begin{center}
\textsc{Solutions}
\end{center}
\end{Large}
\begin{large}
\textsc{Conferences/Workshops\\}
\end{large}
\begin{large}
\textsc{Web-accessibility\\}
\end{large}
\begin{large}
\textsc{Talks\\}
\end{large}
\begin{large}
\textsc{Education\\}
\end{large}
\begin{large}
\textsc{Research\\}
\end{large}
\end{minipage}
\begin{minipage}{0.25\linewidth}
\begin{Large}
\begin{center}
Wrap-up
\end{center}
\end{Large}
\end{minipage}
\end{document}
