\documentclass{beamer}
\usetheme{Pittsburgh}
\usepackage{xcolor}
\definecolor{USUBlue}{RGB}{26,57,89}
\usecolortheme[named=USUBlue]{structure}
\setbeamercolor{normal text}{fg=USUBlue}
\usepackage[utf8]{inputenc}
\usepackage{mathptmx}
\usepackage{tgbonum}
\usepackage{amssymb}
\usepackage{graphicx}
\usepackage{marvosym}
\usepackage[absolute, overlay]{textpos}
\usefonttheme{structuresmallcapsserif}
\usefonttheme{serif}
\setbeamercolor{author}{fg=USUBlue}
\setbeamerfont{author}{size=\small}
\setbeamerfont{frametitle}{size=\large}
\setbeamertemplate{enumerate items}[default]
\author{Elita Baldridge, Department of Biology and the Ecology Center, Utah State University, Logan, UT, USA, @elitabaldridge\\Ethan White, Department of Wildlife Ecology \& Conservation and the Informatics Institute, University of Florida, Gainesville, FL, USA}
\title[17pt]{Ecologist in silico: Facilitating access for chronically ill/disabled ecologists}
\date{}
\setbeamertemplate{navigation symbols}{}

\usepackage[orientation=landscape,size=a0,scale=1.4,debug]{beamerposter}
 
\begin{document}
\begin{flushright}

\begin{center} 
\begin{huge}
\rule{\linewidth}{3cm}
\textsc{%textsc makes text small caps
\\Ecologist in silico: Facilitating access for chronically ill/disabled ecologists\\
 }
\end{huge}  
\begin{large}
\textsc{Elita Baldridge}, Department of Biology and the Ecology Center, Utah State University, Logan, UT, USA, @elitabaldridge\\  
\textsc{Ethan White}, Department of Wildlife Ecology \& Conservation and the Informatics Institute, University of Florida, Gainesville, FL, USA, @ethanwhite\\
\end{large}
\end{center}
\rule{\linewidth}{0.25cm}
\vspace{10cm}
\begin{minipage}{0.25\linewidth}
\begin{Large}
\vspace{0.5cm}
\textsc{Background}\\
\end{Large}
Members of under-represented groups face unconscious and conscious biases which create societal barriers to doing science.  Chronically ill/disabled scientists in particular often face physical as well as societal barriers.\\ 

\textsc{Disabilities can be visible or invisible, mental or physical, present with constant severity, get worse over time, or fluctuate from bad to less bad.\\ } 

I present recommendations to encourage ecology to become a more accessible discipline to those with chronic illness/disability, from a general perspective and on the scale of individual collaborations.\\

\begin{Large}
\textsc{The Default}
\begin{center}
Inaccessibility.
\end{center}
\end{Large}
\begin{Large}
\textsc{The Minimum}\\
\end{Large}
Provide accessibility information \textsc{without} request.\\
\begin{center}
\includegraphics[scale=1]{../sad-comparison/sad-data/dissertation/trinket-access-generator.png}\\
\begin{small}
https://trinket.io/python\\
https://github.com/embaldridge/accessibility-statement-generator\\
\end{small}
\end{center}
What accessibility accommodations are \textsc{and} are not available?  It takes little time on the part of an organizer, but it helps let fellow chronically ill/disabled scientists know that their presence is not an afterthought, and it saves a lot of time and heartache trying (and often failing) to get accommodations after the fact.
\end{minipage}
\begin{minipage}{0.42\linewidth}
\begin{Large}
\begin{center}
\textsc{Solutions}
\end{center}
\end{Large}
\begin{Large}
\textsc{Conferences/Workshops\\}
\end{Large}
\begin{Large}
\textsc{Web-accessibility\\}
\end{Large}
\begin{Large}
\textsc{Talks\\}
\end{Large}
\begin{Large}
\textsc{Education\\}
\end{Large}
\begin{Large}
\textsc{Research\\}
\end{Large}
\end{minipage}
\begin{minipage}{0.25\linewidth}
\begin{Large}
\textsc{Additional Reading}
\end{Large}
\end{minipage}
\end{flushright}
\end{document}
