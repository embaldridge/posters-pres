\documentclass{beamer}
\usetheme{Pittsburgh}
\usepackage{xcolor}
\definecolor{USUBlue}{RGB}{26,57,89}
\usecolortheme[named=USUBlue]{structure}
\setbeamercolor{normal text}{fg=USUBlue}
\usepackage[utf8]{inputenc}
\usepackage{mathptmx}
\usepackage{tgbonum}
\usepackage{amssymb}
\usepackage{graphicx}
\usepackage{marvosym}
\usepackage[absolute, overlay]{textpos}
\usefonttheme{structuresmallcapsserif}
\usefonttheme{serif}
\setbeamercolor{author}{fg=USUBlue}
\setbeamerfont{author}{size=\small}
\setbeamerfont{frametitle}{size=\large}
\setbeamertemplate{enumerate items}[default]
\author{Elita Baldridge, Department of Biology and the Ecology Center, Utah State University, Logan, UT, USA, @elitabaldridge  \\Ethan White, Department of Wildlife Ecology \& Conservation and the Informatics Institute, University of Florida, Gainesville, FL, USA}
\title[17pt]{Ecologist in silico: Facilitating access for chronically ill/disabled ecologists}
\date{}
\setbeamertemplate{navigation symbols}{}

\usepackage[orientation=landscape,size=a0,scale=1.4,debug]{beamerposter}
 
\begin{document}
\begin{center} 
\begin{huge}
\textsc{%textsc makes text small caps
\\Ecologist in silico: Facilitating access for chronically ill/disabled ecologists\\
 }
\end{huge}  
\begin{large}
\textsc{Elita Baldridge}, Department of Biology and the Ecology Center, Utah State University, Logan, UT, USA, @elitabaldridge\\  
\textsc{Ethan White}, Department of Wildlife Ecology \& Conservation and the Informatics Institute, University of Florida, Gainesville, FL, USA, @ethanwhite\\
\end{large}
\end{center}

\begin{minipage}{0.45\linewidth}
\begin{Large}
\begin{center}
\textsc{Background}
\end{center}
\end{Large}
% Generalize language.
% Condense
One factor that we all have to address as scientists is the Eurocentric, cis-gendered, heterosexual, able-bodied white male origins of our field.  Whether we are members of that group or not, we have to confront the historical legacy that has been left behind in the form of conscious and unconscious biases that directly impact the ability of the STEM fields to attract and retain talented scientists of under-represented groups.  Being able to just focus on doing ecology is a luxury that members of under-represented groups do not have, but chronically ill/disabled scientists often face physical, as well as societal, barriers to pursuing science.\\ 


Increasing accessibility for chronically ill/disabled scientists also improves accessibility for researchers who are unable to travel to events for other reasons (lack of funds, distance, etc.). I present recommendations to encourage ecology to become a more accessible discipline to those with chronic illnesses/disabilities, from a general perspective and on the scale of individual collaborations.\\ 

Disclosing a chronic illness or disability, whether mental or physical, is very difficult for many people, and yet it is the only way to get official support from an organization, as the default position is, in practice, a lack of support.  Additionally, without an official diagnosis, it's not possible to get accommodations officially through the university, although it takes months, and often years to get an official diagnosis with many chronic illnesses.  
\end{minipage}
\begin{minipage}{0.45\linewidth}
\begin{Large}
\begin{center}
\textsc{Solutions}
\end{center}
\end{Large}
%Meetings, Conferences/Workshops
%Web-accessibility
%Talks
%Education
%Research
At a bare minimum, 
\end{minipage}

\end{document}
